\subsection{About}\label{about}

CircleCI status:

\href{https://circleci.com/gh/swag-kaust/TD}{\includegraphics{https://circleci.com/gh/swag-kaust/TD.svg?style=svg\&circle-token=2bb57e94a999ba7f33afb12bf091751af6bda219}}

ASOFI3D stands for Anisotropic Seismic mOdeling with FInite differences.
This code is a modification of
\href{https://git.scc.kit.edu/GPIAG-Software/SOFI3D/wikis/home}{SOFI3D}
to accomodate orthorhombic anisotropy

\subsection{Obtaining the code}\label{obtaining-the-code}

Get the code from this repo:

\begin{verbatim}
git clone git@github.com:swag-kaust/TD.git
\end{verbatim}

Then switch to the cloned repo directory:

\begin{verbatim}
cd TD
\end{verbatim}

\subsection{Building the code}\label{building-the-code}

The prerequisites for ASOFI3D are:

\begin{itemize}
\tightlist
\item
  C compiler (for example, \texttt{gcc})
\item
  MPI library (for example, \href{https://www.open-mpi.org/}{OpenMPI})
\item
  Make build system (GNU Make is a popular choice)
\end{itemize}

\subsection{Compile with gcc and OpenMPI on
Ubuntu}\label{compile-with-gcc-and-openmpi-on-ubuntu}

On recent Ubuntu versions such as 14.04, 16.04, or 18.04 all
prerequisites can be obtained by the following commands:

\begin{verbatim}
sudo apt-get install gcc make libopenmpi-dev
\end{verbatim}

Then while in the root directory of the code, build the code via command

\begin{verbatim}
make
\end{verbatim}

which compiles the solver and several auxiliary utilities.

\subsection{Example usage}\label{example-usage}

After successful compilation, you can run the code via command

\begin{verbatim}
./run_ASOFI3D.sh np dirname
\end{verbatim}

where \texttt{np} is a number of MPI processes you want to use and
\texttt{dirname} is the directory that contains configuration of the
problem to solve. Parameter \texttt{dirname} is optional and defaults to
\texttt{par}, so that the main configuration file of the solver is
\texttt{par/in\_and\_out/sofi3D.json}.

\subsection{Running the tests}\label{running-the-tests}

To run the tests, Madagascar is an additional prerequisite. Tests are
run via the command

\begin{verbatim}
make test
\end{verbatim}
